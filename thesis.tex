% !TeX TS-program = xelatex
\documentclass[doctor,final,twoside]{BLCUThesis}
% 必选参数: master/doctor
%% master: 硕士论文
%% doctor: 博士论文
% 可选参数: final
%% 未指定时:封面为匿名评审版
%% 指定时: 封面为最终版
% 可选参数: oneside/twoside
%% 未指定时:默认为 oneside
%% oneside: 单面打印版
%% twoside: 双面打印版(每章都出现在右侧,中间加入空白页)

\begin{document}
	\thesisTitle{
		北京语言大学硕士、博士\\ \LaTeX 论文模板
	}{
		Beijing Language and Culture University \\ Master and Doctor \LaTeX ~Thesis Template 
}
	\stuName{孔存良}
	\stuCountry{中国}
	\stuID{xxxxxxx}
	\stuCollege{信息科学学院}
	\stuMajor{语言智能与技术}
	\stuDirection{自然语言生成}
	\supervisorName{}
	\thesisCompleteTime{二〇二二}{十}{十一}
	\integrityStatementAuthor{example-image-golden}
	\integrityStatementSupervisor{example-image-golden}
	\integrityStatementDate{2022}{10}{11}
	
	\pagestyle{plain}
	\pagenumbering{Roman}
	
	\makecover
	\makeIntegrityStatement
	
	\begin{abstract}
		在麻省理工、清北复交等众多国际国内知名大学早已拥有属于自己的 \LaTeX{} 模板的时代背景之下,北京语言大学理应拥有一份像样的论文模板。本文介绍北京语言大学硕士、博士研究生学位论文 \LaTeX{} 模板的设计与使用说明,权当抛砖引玉。
		
		\keywords{北京语言大学;学位论文模板;\LaTeX 排版系统}
	\end{abstract}
	
	\begin{abstract*}
		Under the background of the times when many well-known international and domestic universities such as MIT and Qingbei Fujiao already have their own \LaTeX{} templates, Beijing Language and Culture University should have a decent thesis template. This paper introduces the design and usage of the \LaTeX{} template for BLCU Master/Doctor thesis, which intends to start further discussion on this issue.
		
		\keywords*{CSUST; Bachelor thesis template; \LaTeX{} typesetting system}
	\end{abstract*}

	\makecontents

\setcounter{page}{1}
\pagenumbering{arabic}

\chapter{工作一}
\section{背景与意义}
\subsection{研究背景}
\subsubsection{ABC}
\begin{figure}[h]
	\centering
	\includegraphics[width=9.8cm]{figures/BLCULogoText.png}
	\caption{北语Logo}
\end{figure}
\begin{equation}
	E = mc^2
\end{equation}

\begin{figure}[h]
	\centering
	\includegraphics[width=9.8cm]{figures/BLCULogoText.png}
	\caption{北语Logo}
\end{figure}

\chapter{工作二}
\section{背景与意义}
\subsection{研究背景}
\subsubsection{ABC}
\begin{figure}[h]
	\centering
	\includegraphics[width=9.8cm]{figures/BLCULogoText.png}
	\caption{北语Logo}
\end{figure}

\begin{figure}[h]
	\centering
	\includegraphics[width=9.8cm]{figures/BLCULogoText.png}
	\caption{北语Logo}
\end{figure}

\chapter{工作三}
\section{背景与意义}
\subsection{研究背景}
\subsubsection{ABC}
\subsubsection{ABC}
ABC\citet{Xiang:20}

ABC\citep{Xiang:20}

\nocite{*}
\bibliographystyle{gbt7714-numerical}
\addcontentsline{toc}{chapter}{参考文献}
\bibliography{ccl2022-zh}

\appendix

\chapter{图表附录}
\section{图片}
\subsection{啥啥啥啥}
\begin{figure}[h]
	\centering
	\includegraphics[width=9.8cm]{figures/BLCULogoText.png}
	\caption{北语Logo}
\end{figure}

\begin{figure}[h]
	\centering
	\includegraphics[width=9.8cm]{figures/BLCULogoText.png}
	\caption{北语Logo}
\end{figure}

\chapter{ABC}
\section{abc}
\subsection{123}

\end{document}